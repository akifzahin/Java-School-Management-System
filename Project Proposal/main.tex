\documentclass[a4paper,12pt]{article}
%\usepackage[utf8]{inputenc}
\usepackage[english]{babel}
\usepackage[scaled=.98]{courier}
%\usepackage{microtype}
\usepackage{graphicx}
\graphicspath{ {./images/} }
\usepackage{wrapfig}
\usepackage{enumitem}
\usepackage{fancyhdr}
\usepackage{amsmath}
%\usepackage[a4paper,inner=1.7cm, outer=2.7cm, top=2cm, bottom=2cm, bindingoffset=1.2cm]{geometry}
\usepackage{blindtext}
\usepackage{index}
\usepackage{hyperref}
\makeindex

%\renewcommand{\familydefault}{\sfdefault}

\begin{document}

\begin{figure}
    \centering
\includegraphics[width=4.5cm]{logo.jpg}    
\end{figure}

\begin{titlepage} % Suppresses headers and footers on the title page

	\centering % Centre everything on the title page
	
	\scshape % Use small caps for all text on the title page
	
	\vspace*{\baselineskip} % White space at the top of the page
	
	%------------------------------------------------
	%	Title
	%------------------------------------------------
	
	\rule{\textwidth}{1.6pt}\vspace*{-\baselineskip}\vspace*{2pt} % Thick horizontal rule
	\rule{\textwidth}{0.4pt} % Thin horizontal rule
	
	\vspace{0.75\baselineskip} % Whitespace above the title
	
	{\Large \textbf{SCHOOL\\ MANAGEMENT \\ \vspace{0.25cm} SYSTEM}   } % Title
	
	\vspace{0.75\baselineskip} % Whitespace below the title
	
	\rule{\textwidth}{0.4pt}\vspace*{-\baselineskip}\vspace{3.2pt} % Thin horizontal rule
	\rule{\textwidth}{1.6pt} % Thick horizontal rule
	
	\vspace{2\baselineskip} % Whitespace after the title block
	
	%------------------------------------------------
	%	Subtitle
	%------------------------------------------------
	
	%\textbf{\normalsize Lab Instructor - Nazmul Alam Diptu \\ Course Instructor - Tarek Ibne Mizan}% Subtitle or further description
	{\scshape\large  Lab Instructor - Nazmul Alam Diptu \\Course Instructor - Tarek Ibne Mizan \\}
	
        \vspace*{3\baselineskip} % Whitespace under the subtitle
	
	%------------------------------------------------
	%	Editor(s)
	%------------------------------------------------
	
	\large Project By
	
	\vspace{0.5\baselineskip} % Whitespace before the editors
	
	{\scshape\large \textbf{Akif Zahin - 2131865042 \\ Mohammed Aman Bhuiyan - 2131864642 \\ Tabassum Bari - 2132177642 \\} } % Editor list
	
	\vspace{0.5\baselineskip} % Whitespace below the editor list
	
	\textit{\large North South University  } % Editor affiliation
	
	\vfill % Whitespace between editor names and publisher logo
        
        
	%------------------------------------------------
	%	Publisher
	%------------------------------------------------
	
	
	\vspace{0.3\baselineskip} % Whitespace under the publisher logo
	
	2022 % Publication year
	
	{\normalsize CSE215.12L - GROUP 3 } % Publisher

\end{titlepage}

%----------------------------------------------------------------------------------------


\tableofcontents
\vspace{3cm}
\begin{center}
    \large Github Repository - Click \href{https://github.com/akifzahin/CSE215.12L-Group-3-School-Management-System}{\textbf{HERE}} 
\end{center}

\newpage


\section{Introduction}
\enlargethispage{\baselineskip}
In this day and age of modern technology, as we progress towards globalization, the number of people interested to pursue higher education increases day by day. But what is the ultimate solution to keeping track of all of the students that enter an institution? The solution is not keeping record in large binders stack full of pages, but rather a complex but impressive software known as a management system. \par While we are well underway in the 21st century, education institutions such as schools, use an archaic and rather painstaking way of keeping track of everything that happens daily in their institution. The proposed software today aims to help pave a way to make the long and stressful method of record keeping and student integration, into an easy and manageable experience. 

\section{Application Objective}
\enlargethispage{\baselineskip}
\begin{itemize}
    \item Make it easier to enhance student life by providing other tools
    \item Reduce the arduous task of keeping records through an automated system
    \item Implement account system for students and teachers
    \item Give ratings towards the teachers in an anonymous manner
    \item Reduce paper wastage and manual labour
\end{itemize}

\section{Target Customers}
\enlargethispage{\baselineskip}
\begin{itemize}
    \item Students  -  an efficient method for them to create an account where they can view their student information, check their attendance, check their previous results and calculate GPA and check their tuition fees payable. They can also review their teachers and give a summary of what they learnt.
    \item School Teachers - can keep a vast record of every new and existing student in their classes as they can view, add or delete students. They can view their course and section info and also give attendance to their respective sections, assign grades and marks.  
    
\end{itemize}
\pagebreak

\section{Value Proposition}
\enlargethispage{\baselineskip}
This system will eliminate the manual and meticulous method of keeping records in an educational institution like a school. Using the system will enable both teachers and students to accurately keep track while using different tools to further enhance their study. \par Different reports can be assigned by the teachers based on their current grades and attendance alike. The software will play the role of a school data management system and allow users complete jobs involving bulk data management flawlessly and quickly.

\section{Software Description}
\enlargethispage{\baselineskip}
As the user enters they will be greeted with a welcome screen with the option to login or register a new account as a student or teacher. 
\par
Teachers will then be able to:
\begin{itemize}
    \item Add or remove students to their sections
    \item Monitor attendance of respective classes
    \item Assign grades to students
    \item View course and section information
\end{itemize}
Students will be able to:
\begin{itemize}
    \item View their own and their course information
    \item Check their assigned grades and calculate GPA accordingly
    \item Review their tuition fees for the term and make payment
    \item Leave anonymous reviews for teachers and the course
\end{itemize}


\section{Tools and Resources Used}
\enlargethispage{\baselineskip}
\begin{itemize}
    \item Java
    \item Java Swing for Graphical User Interface 
    \item Java File Handling for Information Storage
\end{itemize}
\pagebreak

\section{Challenges for Project}
\enlargethispage{\baselineskip}
The method of storing data efficiently and system integration of different modules in a robust manner can be troublesome since there is the factor of project complexity.
\par
Distributing the software into multiple schools can be an issue since many school coordinators may choose to omit the paperless method since they need to integrate a complicated system. 

\section{Improvements and Plans for Future}
\enlargethispage{\baselineskip}
As this program is a small scale system intended for smaller educational institutions such as schools, we plan to expand the software so that larger institutions such as universities or colleges can use this program for a multitude of students incoming that help both the professors and students not only for record keeping, but other features that will enhance the experience of users as a whole. \par The system still needs adequate implementation of advanced front end languages to enhance the user experience and a dedicated database that automatically stores user information. We are hoping to implement these changes as soon as we can so that the system can be more dynamic and robust in the future.
\\ \\


\end{document}
